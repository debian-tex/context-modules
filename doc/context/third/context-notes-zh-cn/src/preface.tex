\title{自序}

\useURL[王垠][https://www.yinwang.org]%[][\hyphenatedurl{https://www.yinwang.org}]

我最早听闻并接触 \CONTEXT,大概是 2008 年在王垠\footnote{一个敢于自我否定的人,曾致力于在国内推广 GNU/Linux 和 \TEX,后来以其对计算机科学和编程语言设计的洞察而闻名,详见其个人主页 \bluebox{\from[王垠]}。}的个人主页上看到他对 \CONTEXT 的谥美之文。当时我猎奇心甚为严重,又觉得大家都在用的 \LaTeX\ 无法体现我的气质,便开始自学 \CONTEXT。或许这是年青人的通病。

现在,我已不再年青,却依然喜欢 \CONTEXT。曾经沧海难为水,或许这是人老了之后的通病,而我觉得更可能是因为 \CONTEXT\ 依然年青。2008 年,\CONTEXT\ 正处于从 Mk\Romannumerals{2} 版本向 Mk\Romannumerals{4} 版本跃迁期间。待 2019 年 MK\Romannumerals{4} 版本尘埃落定时,\CONTEXT\ 的开发者大刀阔斧,如火如荼,开启了下一个版本 \LMTX\cite[Hagen2019]的开发工作,至今方兴未艾。

任何工具,只要有人长时间维持和改进,便会趋向于复杂而难以被他人驾驭,但是它能胜任复杂的任务。\TEX\ 是复杂的,以它为基础的 Plain \TeX,\LaTeX\ 和 \CONTEXT\ 则更为复杂。事实上,并非工具趋向于变得复杂,而是任务趋向于变得复杂,更本质一些,是人心趋向于变得复杂。例如,使用 Markdown 之类的标记语言写散文类的文章,轻松愉快,这是近年来 Markdown 广泛用于网文创作的主要原因。倘若用 Markdown 写一本含有许多插图、数学公式、表格、参考文献等内容的书籍,便需要为它增加许多功能,最终的结果相当于又重新发明了一次 \TeX。

Plain \TeX,\LaTeX\ 和 \CONTEXT\ 虽然皆为构建在 \TeX\ 系统上的宏包,但是国内熟悉前两者的人数远多于 \CONTEXT,究其原因,我觉得是因为 \CONTEXT\ 入门文档甚少,其中中文文档则更为罕见。\CONTEXT\ 创始人兼主要开发者 Hans Hagen 为 \CONTEXT\ 撰写了一份内容全面、排版精美的英文版入门文档\cite[ma-cb-en],对于具备一些英文阅读能力的人,原本可从该文档入门,但遗憾的是,除非读者知道如何在 \CONTEXT\ 中使用汉字(也包括日、韩文字)字体,否则所学知识仅能用于英文排版。

\useURL[ctan][https://ctan.org]
我曾于 2009 年写过一份 \CONTEXT 学习笔记,介绍了在 \CONTEXT\ Mk\Romannumerals{4} 中如何加载汉字字体以及基本的 \CONTEXT\ 排版命令的用法,内容颇为粗陋,由 CTeX 论坛里的朋友整合至 ctex-doc 项目并打包呈交 \bluebox{\from[ctan]} 网站。2011 年夏天曾许诺将我发布于网络上的一些相关文章合并至该笔记,但因俗务缠身,后来兴趣又有漂移,便不了了之。

光阴荏苒,时过境迁,当年曾在 CTeX 论坛一起折腾 \TeX\ 的朋友大多已不知所踪——在他们看来,我亦如是。2018 年,CTeX 论坛因国内日益严厉的互联网监管政策被迫无限期关闭,导致国内 \TeX\ 学习、研究和使用热情似乎遭受了毁灭性打击,爱好者们离散江湖,白头宫女在,闲坐说玄宗。现在,若学习 \LaTeX\ 尚能找到一些讨论区,而 \CONTEXT\ 似乎再也无人问津了。爱好终归属于自己。即使现在国内只有我一个人还在喜欢 \CONTEXT,依然应当为自己写一份新的 \CONTEXT\ 学习笔记,以偿旧诺。

值此情境,需篡改古词一阙,歌以咏志:芦叶满汀洲,寒沙带浅流,二十年重过南楼。欲买桂花同载酒,终不负,少年游。

\vfill
\startalignment[flushright,broad]
2023 年 3 月写于山东郯城乡下老家
\stopalignment